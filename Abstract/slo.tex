\begin{sloabstract}
  \addchaptertocentry{\sloabstractname} % Add the abstract to the table of contents
%  
%   Opravili smo obsežen pregled obstoječih publikacij in primerov uporabe.
%   Publikacije smo prikazali v tabeli, ki je razkrila vrzeli profilov, ki še niso bili raziskani ali uporabljeni.
%   Nato smo analizirali profile obremenitve in s pomočjo t-SNE predstavili, kako so profili povezani v visokodimenzionalnem prostoru. 
%   Z novo prodobljenim znanjem smo razvili sistem za oskrbo starejših oseb, ki jim lahko pomaga podaljšati samosotjno bivanje.
%   Sistem preko analize profilov zazna anomalije v porabi električne energije, ki so pri starejših lahko posledica padcev, kapi ali spremenjenega vedenja zaradi demence.
  
%   Z uspešno implementacijo sistema za oskrbo starejših smo potrdili, da so do sedaj neuporabljeni profili lahko uporabni.
%   Ugotovitve te magistrske naloge prikazujejo neizkoriščen potencial podatkov o energiji, kjer tabela profilov predstavlja osnovo za nadaljnje raziskave na tem področju.
    

% Poraba električne energije in njena optimizacija postajata v današnjem svetu vse pomembnejša, zlasti zaradi vse večje zaskrbljenosti zaradi podnebnih sprememb in staranja prebivalstva. V tej diplomski nalogi so raziskane možnosti podatkov o električni energiji in uporabe profilov obremenitve za reševanje teh vprašanj. Profili obremenitve ali grafične predstavitve značilnosti porabe energije skozi čas lahko pomagajo pri optimizaciji vzorcev porabe električne energije in razvoju rešitev za oskrbo starejših. Učinkovita predstavitev podatkov o energiji s pomočjo profilov obremenitve je stalna zgodba celotne diplomske naloge. S poudarkom na stanovanjskem sektorju, ki predstavlja tretjino porabe električne energije v EU, lahko optimizacija vzorcev porabe bistveno zmanjša človeški odtis.

% Izveden je bil obsežen pregled obstoječih publikacij in primerov uporabe, da bi ugotovili vrzeli v raziskavah in uporabi profilov obremenitve. Te publikacije so bile prikazane v preglednici, ki je razkrila vrzeli v obremenitvenih profilih, ki še niso bile raziskane ali uporabljene. Nato so bili analizirani profili obremenitve in z uporabo t-SNE je bilo prikazano, kako so profili povezani v visokodimenzionalnem prostoru. To je omogočilo globlje razumevanje povezav med vzorci porabe energije in možnimi aplikacijami profilov obremenitve.

% Ena od glavnih aplikacij profilov obremenitve, raziskanih v tej diplomski nalogi, je razvoj sistema za pomoč starejšim. Ta sistem je zasnovan tako, da zazna nepravilnosti v vzorcih uporabe starejših, kot so padci, kapi ali spremenjeno vedenje, ki ga povzroča demenca. Z ugotavljanjem teh nepravilnosti lahko sistem zagotovi pravočasno pomoč in oskrbo za starejšo populacijo. Ta uporaba profilov obremenitve kaže na njihov potencial za reševanje izzivov, s katerimi se sooča starajoče se prebivalstvo.

% v naših raziskavah smo uspešno razvili sistem za odkrivanje anomalij v oskrbi starejših, ki lahko odkrije anomalije z do 80-odstotno učinkovitostjo. Ta izjemna ugotovitev dokazuje, da je človeška rutina dovolj periodična, da omogoča takšno odkrivanje. Da bi dodatno potrdili obstoj te rutine, smo izračunali korelacijo med zemljevidi t-SNE, ki naj bi zaradi svoje podobnosti vključevala moč rutine v evklidskih razdaljah med skupinami. Ta pristop poudarja učinkovitost uporabe profilov obremenitve za prepoznavanje in analizo vzorcev človeškega vedenja, kar na koncu privede do izboljšanja energetske učinkovitosti in dobrega počutja starejše populacije.

% Profili obremenitve se lahko razvrstijo v tri razrede glede na primere uporabe: upravljanje omrežja, odkrivanje anomalij in druge aplikacije. Upravljanje omrežja vključuje preučevanje vzorcev uporabe potrošnikov za izboljšanje energetske učinkovitosti, upravljanje porazdeljenih virov energije in zmanjšanje obremenitve električnega omrežja. Odkrivanje nepravilnosti se osredotoča na prepoznavanje nepravilnosti v vzorcih porabe energije, kot so okvare naprav ali nesreče starejših oseb. Drugi načini uporabe profilov obremenitve vključujejo zaznavanje zasedenosti, povratne informacije o razvoju in preprečevanje kraje energije.

% Za izdelavo profilov obremenitve so potrebni podatki časovnih vrst, ki vsebujejo informacije o porabi energije. V tem delu je bilo uporabljenih pet podatkovnih nizov (UK-DALE, REFIT, ECO, REDD in iAWE), ki so merili porabo električne energije v stanovanjskih stavbah. Zbirke podatkov so bile ponovno vzorčene in narezane na enourne intervale za lažje ravnanje in analizo.

% Prispevki te diplomske naloge vključujejo pregled najsodobnejših profilov obremenitve, razvoj večdimenzionalnih profilov aktivacijske obremenitve, raziskovalno analizo podatkov z uporabo t-SNE in predlog nove metode za odkrivanje anomalij pri oskrbi starejših. S temi prispevki disertacija prikazuje neizkoriščene možnosti energetskih podatkov in zagotavlja podlago za nadaljnje raziskave na tem področju. Z učinkovitim pridobivanjem informacij iz razpoložljivih podatkov in uporabo profilov obremenitve lahko raziskovalci odkrijejo nova spoznanja in razvijejo inovativne rešitve za energetsko optimizacijo in upravljanje oskrbe starejših.

% \section{Uvod}

V tej magistrski nalogi raziščemo možnost uporabe profilov porabe električne energije za naslavljanje ovir samostojnega bivanja starejšega prebivalstva. Osrednja tema magistrske naloge je učinkovita predstavitev podatkov s pomočjo profilov porabe. Optimizacija porabe energije lahko bistveno zmanjša ogljični odtis človeka, saj se v Evropski uniji tretjina električne energije porabi v gospodinjskem sektorju. 

Podnebne spremembe zahtevajo prehod na obnovljive vire energije in prestrukturiranje energetske industrije. V Evropski uniji se tretjina energije porabi v gospodinjstvih, zato je zmanjšanje porabe v tem sektorju ključnega pomena za zmanjšanje človekovega ogljičnega odtisa. Evropska unija si prizadeva biti do leta 2050 podnebno nevtralna, zato išče načine za izboljšanje učinkovitosti vseh onesnaževalcev skozi Evropski zeleni dogovor. Velik del k onesnaževanju prispeva energetski sektor, kjer bi z analizo porabe energije lahko naredili veliko izboljšav, ki bi pomagale doseči zastavljeni cilj. 

Druga težava, s katero se srečujemo, je staranje prebivalstva. Podaljševanje samostojnega življenja starejših posameznikov bi lahko izboljšal dobrobit starejših občanov ter zmanjšal pritisk na zdravstvene in socialne storitve. Z analizo profilov porabe električne energije lahko zaznamo razne nenavadnosti v vzorcih porabe, ki so lahko posledica padcev, kapi ali spremenjenega vedenja zaradi demence. Ob takšnih zaznavi spremenjenih vzorcev vedenja lahko pravočasno okrepimo in poskrbimo za starejše posameznike, ki živijo sami.  

Definicija profila porabe je grafični prikaz porabe energije v določenem časovnem obdobju, ki nam omogoča analizo in vpogled v navade potrošnje. Takšni profili se navadno uporabljajo za analizo porabe električne energije, vendar bi jih lahko uporabljali tudi za analizo drugih virov energije, kot so plin, nafta ali celo voda. Za prikaz magnitude porabe se navadno uporablja moč ($P$) ali energijo ($E$) v nekem časovnem obdobju, najpogosteje čez dan. V tem delu smo se osredotočili na aktivacijske profile porabe. Aktivacijski profili predstavljajo časovne odseke, v katerih poraba presega določeno mejno moč. Za pridobitev teh profilov uporabljamo različne metode obdelave podatkov. Najpreprostejši pristop je določitev mejne moči aktivacije, pri čemer se šteje, da je aktivacija prisotna v vsakem časovnem odseku, ko poraba preseže to mejo. V samem delu smo uporabili bolj kompleksno metodo, katero smo tudi opisali v Poglavju \ref{chapter3}. 

Različne vrste profilov porabe se lahko uporabijo za različne namene. Nekateri profili so osnovani na dnevnih, tedenskih, mesečnih ali letnih časovnih intervalih in zavzemajo porabo celotnega gospodinjstva, spet drugi pa se osredotočajo na specifične gospodinjske naprave. Poleg dvodimenzionalnega prikaza s pomočjo grafov lahko iste podatke predstavimo v tridimenzionalnem prostoru s pomočjo toplotnih kart, kjer toplota predstavlja magnitudo aktivacij oziroma porabe v nekem časovnem obdobju.  

Poleg analize porabe, lahko profile uporabimo za analizo proizvodnje energije, na primer fotovoltaičnih sistemov ali vetrnih elektrarn. V primeru domačih sončnih elektrarn bi nam takšna predstavitev omogočala optimalno uskladitev porabe s potrošnjo, tako bi lahko zmanjšali obremenitve na omrežja. Takšne aplikacije profilov so številne in segajo od gospodinjstva in industrije do upravljanja z omrežjem in odkrivanja anomalij v porabi. Takšne aplikacije profilov porabe in spremljanje naših navad bi pripomogle k bolj učinkoviti potrošnji, kjer bi majhno zmanjšanje porabe na veliki ravni lahko pripomoglo k opaznemu zmanjšanju ogljičnega odtisa. 

V sklopu poglavja \ref{chapter2} smo naredili pregled obstoječih raziskav in primerov uporabe profilov porabe. Za lažji pregled trenutnega dela smo raziskave predstavili v preglednici profilov, ki nam je razkrilila profile, ki še niso bili raziskani ali uporabljeni. Vrzeli so nam omogočile razvoj novih, prej ne uporabljenih aktivacijskih profilov.  

Za razvoj in izdelavo profilov so potrebni podatki v obliki časovnih vrst, ki vsebujejo informacije o porabi energije. Uporabili smo pet podatkovnih zbirk (UK-DALE, REFIT, ECO, REDD in iAWE), ki so merili porabo električne energije v gospodinjstvih. Skupaj smo imeli podatke 30 gospodinjstev, vsa so se nahajala na geografskem prostoru Evrope, razen enega, ki se je nahajal v Aziji, bolj natančno v Indiji.  
Poleg podatkov glavnega števca so podatki vsebovali informacijo o porabi vseh naprav znotraj gospodinjstva, kar nam je omogočalo razvoj profilov porabe za specifične naprave. Zbirke podatkov so bile ponovno vzorčene in razdeljene na enourne intervale za lažje ravnanje in analizo.  

V poglavju \ref{chapter4} smo se osredotočili na predstavitev in analizo obstoječih profilov s poudarkom na primerih uporabe. Poglobljeno znanje nam je omogočalo predstavitev in umestitev novih aktivacijskih profilov.  

V poglavju \ref{chapter5} smo uporabili enega izmed novih trodimenzionalnih profilov skupaj z metodo t-SNE (t-distributed stochastic neighbor embedding). To je nelinearna metoda zmanjševanja dimenzij visoko dimenzijskih podatkov. Rezultat te metode je nizko dimenzionalen zemljevid, po navadi dvodimenzionalen, kjer so skupaj vzorci, ki so si podobni v visoko dimenzionalnem prostoru. V primeru linearnih metod za zmanjševanje dimenzij, kot je PCA, lahko nizkim dimenzijiam pripisujemo določene lastnosti, med tem ko pri nelinearnih metodah nizke dimenzije nimajo posebnega pomena. Pomembna je le razdalja med točkami, saj ta predstavlja podobnost vzorcev. Podobni vzorci po navadi formirajo gruče, le te pa lahko uporabimo za analizo navad porabe. S pomočjo metode t-SNE smo zgradili zemljevid naprav, le ta pa omogoča boljše razumevanje značilnosti in vzorcev porabe med različnimi napravami. Med drugim smo prepoznali, da je v profilih porabe znotraj gospodinjstev prisotna rutina. Razdalja med vzorci na zemljevidu predstavlja jakost rutine. bolj kot so vzorci skupaj, bolj močna je rutina določenega gospodinjstva. 

Prisotnost rutine znotraj gospodinjstev pomeni, da lahko odkrivamo razna odstopanja v porabi. Bolj kot je bila rutina močna v preteklosti, bolj smo lahko prepričani, da gre za nenavaden dogodek v primeru nenavadnega vzorca. Primerna ciljna publika za tako metodo so starejši posamezniki, ki živijo sami. Za podaljšanje njihove samostojnosti, obenem pa ohranjanje njihove varnosti lahko analiziramo njihove vzorce porabe in v primeru anomalije obvestimo negovalca o potencialni težavi, kot je recimo padec. Starejši imajo po navadi bolj ustaljene vzorce porabe, kar je za tak sistem zelo primerno. Pasivni sistem morda ni tako hiter pri odzivanju, kot pametne zapestnice, ki bi padce zaznale takoj. Pasivni sistem ima ključno prednost v tem, da deluje neprekinjeno in ne invazivno. V primerih, kjer bi uporabili samo en števec, je tak sistem tudi veliko cenejši, saj bi lahko uporabili že obstoječo infrastrukturo. Za razvoj sistema smo uporabili enega izmed prej še ne raziskanih aktivacijskih profilov porabe, ki smo ga identificirali z uporabo preglednice. Sama metoda je bolj natančno opisana v Poglavju \ref{chapter6}. Le-ta temelji na principu zaznavanju aktivnosti, kot so priprava zajtrka, kosila ali gledanje televizije.  

Čeprav nismo imeli podatkov z dejanskimi anomalijami, smo ugotovili, da je rutina povpreč-nega gospodinjstva dovolj visoka za uporabo zgoraj omenjenega sistema. Rezultati kažejo, da je rutina povprečnega gospodinjstva znotraj določenih časovnih obdobij več kot 80 \%. To pomeni, da bi bil vsak peti vzorec označen kot lažno pozitiven. Sistem smo prilagodili za rabo v dejanskih okoliščinah, z izgradnjo iterativnega učenja, kjer sistem lahko zaživi že po enem mesecu uporabe in se z časom izboljšuje. Pomembna ugotovitev je, da ima povprečno gospodinjstvo dovolj močno rutino za detekcijo anomalij. Ob predpostavki, da je rutina starejših še višja, bi se s tem dodatno izboljšala zanesljivost. Nazadnje smo izračunali korelacijo med rutino in evklidsko razdaljo med vzorci na t-SNE zemljevidu. Čeprav so se uporabile druge metode in drugi profili, je bila korelacija skoraj 80 \%. Rezultat nakazuje, da razdalja vsebuje informacijo o rutini, kar pomeni, da bi lahko uporabili metodo t-SNE za podobne aplikacije.  

Prihodnje raziskave na tem področju ponujajo veliko odprtih možnosti za izboljšavo trenutnega pristopa. Za bolj točne rezultate bi se lahko poslužili metode, kjer bi modelirali različne sintetične anomalije in jih vstavili v obstoječe podatke. Dodatno bi sistem lahko postavili okolje, kjer živjo starejši posamezniki. Realne anomalije, bi nam omogočile dokončno evaluacijo sistema. Obstajajo podobne metode, zato bi bilo bistvenega pomena primerjati točnost naše metode z ostalimi publikacijami. Dodatno bi lahko primerjali točnost metod, kjer se za detekcijo padcev uporabljajo senzorji, kot so pametene zapestnice.  

Ugotovili smo, da lahko z metodo t-SNE zaznamo prisotnost rutine, kar bi nam posledično omogočalo zaznavo anomalij, podobno kot pri našem sistemu razvitem v poglavju \ref{chapter6}, kjer smo uporabili podatke števcev vsake posamezne naprave. Prednost te metode je, da smo uporabili podatke glavnega števca, kar bistveno poveča uporabnost tega pristopa. Enostavnost namestitve sistema je bistvenega pomena za uspešno uvajanje novih tehnologij. V tej luči bi bila evalvacija in razvoj takšnega sistema visokega pomena in ena izmed možnih smeri prihodnjih raziskav. Poleg t-SNE bi bilo zanimivo preveriti ostale metode zmanjševanja dimenzij, kot so PCA ali bolj napredna UMAP metoda. Uporaba tako razvitih, kot tudi novih profilov s temi metodami bi lahko izboljšala trenutni pristop merjenja jakosti rutine ali celo razkrila nova področja uporabe.  

Menimo, da je naše delo prispevalo nova orodja za razumevanje in odkrivanje anomalij v porabi električne enegije, kjer sta energetska učinkovitost in podaljševanje samostojnosti starejših sta le dva primera uporabe, ki smo jih naslovili. Še vedno se je treba veliko naučiti o tem, kako in katere LP lahko uporabimo za izboljšavo kakovosti našega življenja. Medtem, ko smo v preglednici profilov zapolnili nekaj vrzeli, prepuščamo znanstveni skupnosti, da zapolni preostale.  


\par\textbf{Ključne besede}: \slokeywordnames 
  
  \end{sloabstract}