\begin{sloabstract}
  \addchaptertocentry{\sloabstractname} % Add the abstract to the table of contents
  V tem delu raziščemo možnost uporabe profilov porabe električne energije za naslavljanje ovir samostojnega bivanja starejšega prebivalstva . 
  Osrednja tema magistrske naloge je učinkovita predstavitev podatkov s pomočjo profilov porabe.
  Optimizacija porabe energije lahko bistveno zmanjša ogljični odtis človeka, saj se v Evropski uniji tretjina električne energije porabi v gospod sektorju.
  
  Opravili smo obsežen pregled obstoječih publikacij in primerov uporabe.
  Publikacije smo prikazali v tabeli, ki je razkrila vrzeli profilov, ki še niso bili raziskani ali uporabljeni.
  Nato smo analizirali profile obremenitve in s pomočjo t-SNE predstavili, kako so profili povezani v visokodimenzionalnem prostoru. 
  Z novo prodobljenim znanjem smo razvili sistem za oskrbo starejših oseb, ki jim lahko pomaga podaljšati samosotjno bivanje.
  Sistem preko analize profilov zazna anomalije v porabi električne energije, ki so pri starejših lahko posledica padcev, kapi ali spremenjenega vedenja zaradi demence.
  
  Z uspešno implementacijo sistema za oskrbo starejših smo potrdili, da so do sedaj neuporabljeni profili lahko uporabni.
  Ugotovitve te magistrske naloge prikazujejo neizkoriščen potencial podatkov o energiji, kjer tabela profilov predstavlja osnovo za nadaljnje raziskave na tem področju.
    
  
    \par\textbf{Ključne besede}: \slokeywordnames 
  
  \end{sloabstract}