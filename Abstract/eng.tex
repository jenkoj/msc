\begin{abstract}
    \addchaptertocentry{\abstractname} % Add the abstract to the table of contents
    This work explores the potential of electrical energy data
    and how load profiles can be used to address issues such as the optimization of electrical energy consumption patterns and the aging population.
    The efficient presentation of energy data through load profiles is a constant narrative throughout the thesis.
    Optimizing consumption has the potential to significantly reduce the human footprint
    since a third of electrical energy in the EU is consumed in the residential sector.
    Furthermore, we utilize load profiles to address issues such as the aging population. 
    We developed an elderly care assisted living system to detect anomalies in the usage patterns of the elderly.
    The system identifies accidents such as falls, strokes, or dementia-induced altered behavior.

    We performed a comprehensive review of existing publications and use-cases.
    These publications were mapped into a table, which revealed gaps in the load profiles that were not yet researched or used.
    Next, we analyzed the load profiles and using t-SNE presented how profiles are related in high dimensional space. 

    With the successful implementation of the elderly care system, we confirmed that unused load profiles are applicable.
    The findings of this thesis showcase the untapped potential of energy data where the table of profiles provides a foundation for further research in this area.

    \par
    \par\textbf{Keywords}: \keywordnames 

\end{abstract}
