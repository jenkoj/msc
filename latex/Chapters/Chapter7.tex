% Chapter Template

\chapter{Elderly care demo} % Main chapter title

\label{Chapter7} % Change X to a consecutive number; for referencing this chapter elsewhere, use \ref{ChapterX}

%----------------------------------------------------------------------------------------
%	SECTION 1
%----------------------------------------------------------------------------------------
The chapter will first address the issue being solved, then the methodolgy will be presented and finaly the results will be presented.
Elderly care has been addressed by many EU funded research projects since aging population is one of the issues the union is facing. 
There are many solutions for this problem, from invasive such as wearables, sound sensors, IR occupancy detector, etc. 
Few papers tried to solve this issue using non-invasive using previously mentioned NILM algorithms. 
In this case no additional meters need to be installed, since per-appliance usage can be disagreed from main meter time-series data. 
While this is practical from new equipment side, it is a bit less from accuracy side for larger buildings. 
There is a middle way between invasive and non-invasive ways. 
It is possible to set up sub-meters behind each appliance, and indirectly observe the usage pattern for this elder person. 
This is better since the elder does not need to wear the device, but it is a bit slower in detecting the accident. 

\section{Goal}

The chapter will focus on building the elderly care system that will use users periodic usage pattern to detect an anomaly.
The anomaly could be anything from a fall, stroke or altered usage pattern due to dementia. 
Algorithm will be designed based on load profile from previous chapter \ref{fig:PHPA}.
Figure shows, that first thing in the morning used are kettle and toaster, and with delay of one hour, microwave and TV. 
If none of these appliances are used within that hour, then that hour is considered anomalous.
This means that the algorithm will be able to detect the anomaly within 1 hour of accident.

\section{Methodology}

\subsection{Steps to build a load profile}
\subsubsection{Step one}
To detect the anomalies one first needs to build a daily activation profile for each appliance, such as the one previously mentioned \ref{fig:PHPA}.
In this specific case we will be using 2h buckets, yielding total of 12 buckets. 

\subsubsection{Step two}
Find and ignore appliances that are always on by calculating the standard deviation of activations for each bucket.
During this, the fridge should get ignored since its normal distribution is quite narrow compared to other more dynamic appliances.

\subsection{}


%-----------------------------------
%	SUBSECTION 1
%-----------------------------------
\subsection{Subsection 1}

