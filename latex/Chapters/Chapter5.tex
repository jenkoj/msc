% Chapter Template

\chapter{Contributions} % Main chapter title

\label{Chapter5} % Change X to a consecutive number; for referencing this chapter elsewhere, use \ref{ChapterX}

%----------------------------------------------------------------------------------------
%	SECTION 1
%----------------------------------------------------------------------------------------

\section*{introduction}

The goal of this Masters thesis is to propose suitable consumption profiles for supporting residential building consumption optimization and elderly care management.
To achieve this goal, we propose developing activity profiles as follows:. 

The first step is to obtain a set of datasets. In our case this will be UK-DALE, REFIT, ECO, REDD and iAWE.
All datasets measured electrical energy consumption for residential buildings. 
They include main smart meter data, as well as sub-meter data for each appliance in a dwelling. 
For easier handling datasets will be sliced to 1 hour intervals. 
Slices will be put through classical learning classifier in order to demonstrate the ability to automatically identify and classify the appliances.
Data will be then used to generate different daily per appliance usage profiles.

% \begin{figure}[h!]
% 	\centering
% 	\caption{"Universal normalized daily usage profile for weekend and weekday for a microwave. Superposition of data from 25 homes."}
% 	\includegraphics[width=0.9\textwidth]{../Figures/microwave_norm_n25.png}
% 	\label{fig:UniNormMicrowave}
% \end{figure}

One such example can be seen in figure 5. Histogram shows normalized daily 
activation for microwave. It consists of data from 25 homes from 4 different
datasets. 

During the course of this thesis more profiles will be presented, using various histogram buckets, dimensions and parameters. 
Each profile presents data from a different perspective, therefore each profile will have use-case of its own.
Using existing and contributed profiles a map of profiles will be made. 

To demonstrate the usage of developed profiles a demo will be presented. 