\chapter{Conclusion} % Main chapter title

\label{Chapter9} % Change X to a consecutive number; for referencing this chapter elsewhere, use \ref{ChapterX}

In contributions chapter \ref{Chapter5}, it was said that the goal of the thesis will be achieved by answering the three following questions:

\begin{itemize}
	\item How do we efficiently present big data to humans and machines?
	\item How is this presented data connected in a higher dimension?
	\item Can we use one of the profiles to build something useful? 
\end{itemize}

By answering the first question we have found new, previously unused ways of presenting the data.
This was achieved by building a detailed table of profiles such as we have seen in chapter \ref{chapter6}.
This table presented the missing gaps, and which presentations were not used by the community.
We knew that not all unused profiles were useful, by using other publications we have classified them based on their impact. 
We have selected the few with the highest impact and practically used them in the following chapters.

Furthermore, we also presented all the load profiles in high detail. This was done so that the reader was able to understand what the load profiles look like and what they present.
While doing that we pointed out how some profiles could be used, and how we will use them to prove that they are useful. 

The second was answered in chapter \ref{Chapter8}, where we have shown how data is connected in high dimension space
using t-SNE for dimensionality reduction. Here we have shown how some buildings have more similar activation patterns than others.
Furthermore, we have shown which appliances are being used similarly. 
We have grouped the appliances into appliance groups and showed that appliances from different datasets are being used similarly, and how this method and groups can help us label unlabeled data. 
The formed clusters showed that a routine and persistent usage pattern does exist. This laid the groundwork for elderly care, where we have used this routine at the center of the algorithm.

The last was answered in chapter \ref{Chapter7}, by building functioning elderly care assisted living system. 
The results proved that we successfully used one of the proposed load profiles in a real-world scenario. 
As mentioned we have detected a persistent routine. The main goal was to efficiently extract this routine, and build a working system around it.
The results show that we have succeeded in doing that and that the algorithm is adequate to be used in the real world.
To prepare the algorithm for the real world even further, we have implemented an iterative learning system.
The system could be put online a month after the installation of the system and continues to improve over time.

We could go into larger details and use other dimensionality reduction algorithms for comparison, or even have more empirical proofs. 
In the case of elderly care, we could use the results or algorithms of other publications to compare it to our, or even compare it to the other intrusive methods. 
This could all be done in greater detail and even more empirical ways.

In the end, that was not the point.
The point of the two chapters was to prove that the proposed profiles can be efficiently utilized. 
By doing that, we have achieved the main goal of the thesis,
that was "to propose suitable consumption profiles for supporting residential building consumption optimization and elderly care management".
With that, we can conclude the thesis with the following words:

The ever-increasing amount of data is available to the scientific community.
This data is useless if we do not find ways to efficiently extract the information that it is holding.
The sole purpose of the load profiles is to reveal patterns, contextual features and information itself in the vast sea of data.
With the proposed load profiles, we have hopefully contributed new tools that will enable scientists to efficiently uncover the truths that data holds. 
While we have filled in a few gaps in the table of profiles, it is up to scientific commonly to fill in the rest.