\chapter{Introduction}
\label{chapter1}

Climate change calls for a shift to renewable energy and restructuring of the electric power industry.
Sources \cite{eurostat2020} show as of the time of reading this paper, 44 \% of produced electricity in Europe was from combustible sources such as gas, fuel, and coal. Even 
though that is a significant decrease of 10 \% in the last 10 years, it is a significant Co2 emitter.
The same source \cite{eurostat2020} also states that a third of energy is consumed by the residential sector. It is estimated, 
that the human population will reach 10 billion inhabitants in the next 10 years, and ever-increasing ownership of electrical appliances such as smartphones, HVACs, and EVs will further elevate this issue.
Acknowledging this, reducing consumption in the residential sector could leave a significant impact on the human footprint. 

%increased time spent indoors*

The EU aims to be climate neutral by 2050, therefore it seeks to improve the efficiency of every part of pollution contributors through The European Green Deal.
A large part of these contributors is the Energy sector.
A subpart of the energy sector is the residential sector, where many advancements could be made to help to reach the goal.  

This could be achieved through various applications and methods that use load profiling as their core technology.
Authors in \cite{Chuan2014} proposed a method to reduce peak loads by studying consumer
appliance usage patterns. Paper \cite{Csoknyai2019} studied consumer usage patterns, and returned feedback that contributed to reducing consumption.
Another notable way is the use of distributed energy resources and managing them in such a way as to decrease the net output of energy flow such as the authors describe in
\cite{MORENOJARAMILLO2021445}. All described methods would reduce and alleviate the load off the power grid.

Load profiling in building energy consumption is not a novelty and had been in research since the 1980s.
While it was thought that aggregated LPs of households are relatively predictable, recent data obtained using smart meter data showed large deviance from user to user due to different lifestyles, as the author states in \cite{Review2021}.
In recent years LPs have changed due to renewable energy accelerated development of distributed energy resources such as residential photovoltaic
power plants, home wind energy, and using EVs with home batteries. Socioeconomic changes such as work-from-home, also drastically reshaped the LP curve. 

The thesis aims to propose and develop new, previously unused LPs, that will contribute to mitigating the raised issues. 
Before we disclose our contributions, let us first have an overview of what load profiles are and in which other use cases they can be utilized, besides the energy saving that we just mentioned.

% finishing words here 
\subsection{Feature Set} 
\label{sec:feature_set}

If we want to find the base features, we have to look at how consumption measurements are 
done in most buildings. 
The following three features enable us to know the amount of energy being consumed by the user.

\begin{figure}[H]
  \Tree[.base\ features [.power ]
          [.timestamp ]
          [.name ]
                ]
\end{figure}

If we translate these features to the time domain and observe them over a specified amount of time, new features emerge. 
The most notable example is the observation of electrical power over one hour.
The result is energy $E$, and it is one of the most common ways used to bill a customer for his power consumption.


We can also extract features such as the number of activations or time of operation for each activation.
This can be done using sensors to detect activity or even deduct this from power consumption data.
In cases where we are observing individual appliances, this can be done using simple signal processing
techniques. In cases where we are observing buildings, this could be achieved using more complex disaggregation algorithms also known as NILM algorithms.

% Time domain
\begin{figure}[H]
  \Tree[.time\ domain\ features [.energy $E$ ]
          [.number\ of\ activations $N_{act}$  ]
          [.operating\ time $t_{oper}$  ]
                ]
\end{figure}

\begin{figure}[H]
	\centering
	\caption{Simple signal processing of power consumption for a single appliance}
	\includegraphics[width=0.9\textwidth]{Figures/profile_sketches/singal_processing_thr.png}
	\label{fig:sig_proc_fig}
\end{figure}

As we can see in Figure \ref{fig:sig_proc_fig} all three-time domain features can be extracted from the graph. 
Energy $E$ is equal to the area under the graph or in other words integral of power over time. 
$N_{act}$ can be measured based on the number of times the power value exceeded some pre-defined threshold $P_{thr}$. 
The $t_{oper}$ is the time between on and off events, where we use the same threshold as with $N_{act}$.
While there are other features, such as time between activations, or total operational time that could be
extracted, these were not commonly used in related work.

\begin{figure}[H]
\Tree[.frequency\ domain\ features [.Operation\ Modes ]]
\end{figure}
The same as we can present power in the time domain, the same can be done in the frequency domain. 
One actual example can be seen in Figure \ref{fig:freq}.
Here, it is hard to extract more features, but one possibility could be
detecting the number of operation modes based on the number of peaks, using signal processing algorithms.

\begin{figure}[H]
	\centering
	\caption{Frequency of power values for the toaster. Actual data from the REFIT dataset.}
	\includegraphics[width=0.9\textwidth]{Figures/profile_sketches/freq.png}
	\label{fig:freq}
\end{figure}

In the case of Figure \ref{fig:freq} we are observing a toaster over one year.
Toasters are usually simple appliances using a heating element and a thermostat, meaning that the power consumption should be constant and set by the resistance of the heating element.
The Figure shows \ref{fig:freq} a nice normal distribution of power values around 3 kW, 
which we can assume is the heating element.
We can notice two other peaks one at roughly 1 kW and the other at 0.7 kW.
Since toasters usually do not have operation modes, we could assume that there 
are other appliances plugged into the metering device, 
meaning this could be a use-case for this kind of load profile.

\section{Definition and Types of LPs}
\label{sec:LP_types}
Author \cite{Review2021} defines terms as following:

\begin{itemize}
	\item Load: the electricity that all the electricity-powered devices in the household consume in unit time.
	\item Profile: a graph representing the significant features of the electricity load over time.
\end{itemize}


While the most commonly used feature is power, there are other derivatives such as the number of activations of an appliance or operating time.
Usually, the LPs are presented as a daily power consumption profile such as shown in Figure \ref{fig:daily_power_profile}. 
This profile is also known as the standard daily load profile. 
While the LP is a sketch, it still presents consumption trends in morning and evening peaks.

\begin{figure}[H]
	\centering
	\caption{Average daily usage profile for an appliance or a building}
	\includegraphics[width=0.9\textwidth]{Figures/profile_sketches/Slide1.png}
	\label{fig:daily_power_profile}
\end{figure}

Alternatively, we can use a histogram-based presentation such as can be seen in Figure \ref{fig:daily_act_profile}.
While Figure \ref{fig:daily_act_profile} presents the same data as Figure \ref{fig:daily_power_profile},
due to data processing, it could potentially reveal more relevant consumption patterns.

\begin{figure}[H]
	\centering
	\caption{Histogram of daily activations profile for an appliance or a building}
	\includegraphics[width=0.9\textwidth]{Figures/profile_sketches/Slide5.png}
	\label{fig:daily_act_profile}
\end{figure}

LPs can present whole-house usage as well as per-appliance usage, where each presentation has its advantages and disadvantages. 
To present more information, sub-meter data can be used to present whole-house usage with per-appliance contributions.
Two such examples can be seen in Figure \ref{fig:daily_act_m_profile} and \ref{fig:daily_power_m_profile}.

\begin{figure}[H]
	\begin{subfigure}{.5\textwidth}
		% \centering
		\caption{Histogram of daily activations profile for appliance A and B}
		\includegraphics[width=1.1\textwidth]{Figures/profile_sketches/Slide8.png}
		\label{fig:daily_act_m_profile}
	\end{subfigure}%
	~ 
	\begin{subfigure}{.5\textwidth}
		% \centering
		\caption{Average weekday power consumption for appliances A, B and C}
		\includegraphics[width=1.1\textwidth]{Figures/profile_sketches/Slide2.png}
		\label{fig:daily_power_m_profile}
	\end{subfigure}%
	\label{fig:daily_m_profile}
	\caption{LPs with multiple appliances}
\end{figure}

To present as much information as possible,
all the above-mentioned attributes can be presented in a multidimensional way such as shown in Figure \ref{fig:heatmap_2dtime} and \ref{fig:heatmap_all_appl}.

\begin{figure}[H]
	\centering
	\caption{Number of daily activations/power consumption of one appliance/house in one-month period}
	\includegraphics[width=0.9\textwidth]{Figures/profile_sketches/Slide10.png}
	\label{fig:heatmap_2dtime}
\end{figure}

Figure \ref{fig:heatmap_2dtime} is a sketch, and it does not present real-world data.
Even though, it is still possible to see the consumption throughout each day that the plot presents one month of data, where we can see the consumption throughout each day.
The brightness presents the activity of the household or an appliance. 
The brighter the plot, the more activity for that hour of that day of the month.
One other thing to keep in mind when reading a such profile is that the origin is placed in the upper left corner.
This originates from image processing standards.

Alternatively, we can use the heatmap to present the activity of all appliances in a household over a period of time.
Such an example is Figure \ref{fig:heatmap_all_appl}.
In this case, we plot the consumption throughout the day, and it enables us to compare which appliances are being activated together throughout the day.

\begin{figure}[H]
	\centering
	\caption{Consumption for each appliance in a day}
	\includegraphics[width=0.9\textwidth]{Figures/profile_sketches/Slide12.png}
	\label{fig:heatmap_all_appl}
\end{figure}

\section{LP Use-cases}
\label{sec:use_cases_tree}

\begin{figure}[H]
	% \caption{General classification of LP use-cases}
	\label{tree:clasification_of_use_cases}
	\Tree[.{LP Use\ Cases} 
	[.Grid\\Managment Energy\\saving Zero\\Energy\\Buildings Demand\\response ]
	[.Anomaly\\Detection Elderly\\Care Fault\\Detection ]
	[.Other Develo-\\pment\\Feedback Occupancy\\Detection Energy\\Stealing ]
		]
\end{figure}

The load profiling method has a lot of different use cases across different fields.
In our case, we will split use cases into three classes.

The first class is grid management.
For example, it can be used to save energy by studying users' usage patterns and returning feedback, with suggestions on how to improve consumption.
In cases where buildings have grid batteries and PV installed, the same feedback could be used to minimize the amount of energy being pulled from the grid.
These are so-called zero-energy buildings (ZEB).
Electrical energy providers could use demand response programs in combination with the LPs to optimize the management of the grid, with minimal impact on users' daily lives.

The second class is anomaly detection.
The load profiles could be used to help the elderly in case of an accident or even help prevent one. 
They could be used to detect all kinds of early malfunctions in the operation of appliances, which would reduce service costs and save energy.

The last class is other, where occupancy detection, development feedback and energy stealing are all cases where LPs could be used. 

A more detailed description of each use-case with publications will be addressed in the next chapter in section \ref{sec:use-cases}


\section{Contributions}
\label{sec:contributions} 

The main goal of the master's thesis is to propose suitable LPs for supporting residential building consumption optimization and elderly care management.
To achieve this goal, we propose the following steps, where each step is a contribution to the scientific community.

\begin{enumerate}
	\item Surveying the state-of-the-art LPs (\ref{chapter2})
	\item Development of multidimensional activation LPs (ALP's) (\ref{chapter4})
	\item Visual analysis of ALP's (\ref{chapter5})
	\item Propose a new anomaly detection method for elderly care (\ref{chapter6})
\end{enumerate}


The first contribution will be provided by taking a look at existing research and use-cases. 
Using the publications, a table of profiles will be constructed.
The table will provide an overview of existing work, and reveal the gaps with LPs that were not yet utilized.
Using the related work we will try to determine in what field each LP could be used. 
While we will fill some gaps, many will be left open for fellow researchers to pursue.  

The second contribution will be provided in chapter \ref{chapter4}.
Here we will offer an in-depth look into the LPs, by presenting the profiles and showing how they present the consumption patterns.
Each LP presents a different pattern and therefore has a different use case. 

The third contribution will be provided in chapter \ref{chapter5}.
Here, we will use the t-SNE dimensionality reduction algorithm to show how samples are related.
By doing that we will obtain an understanding of the content that the datasets hold.

This newly found knowledge should help us provide the last contribution.
It will be provided in chapter \ref{chapter6}.
Here, we will design and construct elderly care assisted living system by utilizing one of the proposed profiles.
The system will be able to detect anomalies in the daily routine of the elder.
It should be simple, efficient and ready for real-world use.
With that, we should be able to prove that the LP can be efficiently utilized,
thus achieving the main goal of this thesis.