\chapter{Introduction}
\label{chapter1}

Climate change calls for a shift to renewable energy and restructuring of the electricity sector.
Sources \cite{eurostat2020} show as of the time of reading this paper, 44 \% of produced electricity in Europe was from combustible sources such as gas, fuel, and coal. Even 
though that is a significant decrease of 10 \% in the last 10 years, it is a significant Co2 emitter.
The same source \cite{eurostat2020} also states that a third of energy is consumed by the residential sector. It is estimated, 
that the human population will reach 10 billion inhabitants in the next 10 years, and ever-increasing ownership of electrical appliances such as smartphones, HVACs, and EVs will further elevate this issue.
Acknowledging this, reducing consumption in the residential sector could leave a significant impact on the human footprint. 

%increased time spent indoors*

The EU aims to be climate neutral by 2050, therefore it seeks to improve the efficiency of every part of pollution contributors through The European Green Deal.
A large part of these contributors is the Energy sector.
A subpart of the energy sector is the residential sector, where many advancements could be made to help to reach the goal.  

This could be achieved through various applications and methods that use load profiling and load monitoring as their core technology.
Authors in \cite{Chuan2014} proposed a method to reduce peak loads by studying consumer
appliance usage patterns. Paper \cite{Csoknyai2019} studied consumer usage patterns, and returned feedback that contributed to reducing consumption.
Another notable way is the use of distributed energy resources and managing them in such a way as to decrease the net output of energy flow such as the authors describe in
\cite{MORENOJARAMILLO2021445}. All described methods would reduce and alleviate the load off the power grid.

Load profiling in building energy consumption is not a novelty and had been in research since the 1980s.
While it was thought that aggregated load profiles of households are relatively predictable, recent data obtained using smart meter data showed large deviance from user to user due to different lifestyles, as the author states in \cite{Review2021}.
In recent years load profiles have changed due to renewable energy accelerated development of distributed energy resources such as residential photovoltaic
power plants, home wind energy, and using EVs with home batteries. Socioeconomic changes such as work-from-home, also drastically reshaped the load profile curve. 

Technology advancements in non-intrusive load monitoring and increased adoption of smart energy meters offer a new
way of load profiling, that is NILM load profiling.
