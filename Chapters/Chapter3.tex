\label{chapter3}
\chapter{Related work}

Work relating to load profiling can be found in two research verticals or topics. The first one is load profiling and load profile models, which in 
most cases study the load profile curve of the building. Few exceptions study load profiles on appliance-level.
The second vertical is anomaly detection in building energy consumption data. While the first topic is closer, there are quite a few connections with the latter. 
If one wants to do anomaly detection, in some cases, one must first build some kind of "normal consumption profile" 

\section{Load profiling}

One of the first publications on load profiling was published by \cite{TRAIN19851103}.
They used a bottom-up approach using sub-meter data and other socioeconomic and demographic characteristics 
to create a load profile or statistically adjusted engineering (SAE) as they call it.
They can adjust the curve based on weather, dwelling size, and income. 
In the same year, \cite{WALKER1985} published a paper where they used a bottom-up approach with psychological factors to create probability models of when will an individual use an appliance.

Since then there were two more in 1995. Research picked up the pace in 2005 with 7 publications in 2013 as figure \ref{fig:Distribution} shows.

Load-profiling can be performed in two ways: bottom-up and top-down. 

A bottom-up approach as \cite{SWAN20091819} states "calculates the individual dwelling energy or electricity consumption and extrapolate these results over a target area or region"
Whereas with Top-down approach as \cite{SWAN20091819} states "uses the total energy or electricity consumption estimates to assign them to the characteristics of the building stock"
In other more general words, Bottom-up uses sub-meter data, Top-down uses aggregated data. In our case, we take a deeper dive into the bottom-up approach, since it is more relatable.

\cite{Review2021} did a comprehensive review on load profiling. The author defined various load-profile application
subgroups such as demand-side management, planning and control design of energy systems, and residential load profiles. The author also 
grouped modeling techniques as probabilistic models, Markov chains, and Monte Carlo. The author first disclosed the current state of load profiling and issues with past work.
They made a review of existing load profiling models
and asses the-state-of-the art. 
Next, they pointed out future research directions
and applications of load profiling models. Finally, the author exposes issues that researchers face and addresses possible solutions with conclusions.

\begin{figure}[H]
	\centering
	\caption{"Distribution of publications on load profiling from 1985 to 2020. The graph was published by \protect\cite{Review2021}"}
	\includegraphics[width=0.9\textwidth]{Figures/publications.png}
	\label{fig:Distribution}
\end{figure}

\cite{GERBEC2005} tried to assign typical load profiles to a particular group of consumers based on their activity. 
To achieve that, they used probabilistic neural networks as a way of classification. Their methodology was tested in real use scenario. 

\cite{Gao2018} makes use of the bottom-up method to build a forecasting framework for household
load profiling, which takes into account the consumption patterns of residents. 
A model falls into the demand-side management subgroup.
They have developed a "single-day extraction model", designed to select the same days by comparing environmental and household factors, which influence energy consumption.
By using this approach, they have improved the accuracy of predicting behavioral patterns of dwellers. 
This method falls into the probabilistic method subgroup. Results show that their method successfully modeled daily usage.

\cite{Chuan2014} uses load profiling to optimize energy consumption distribution during the day.
This reduces peak usage and alleviates load off the grid. The author used the bottom-up method, that is, using sub-meter data.
Using this data, they made daily usage analyses on a one-hour basis. Using this information they optimized the daily activation of appliances
so that peak usage was not as high. Results show that peak shedding was successful. 

\cite{Csoknyai2019} analyzes energy consumption patterns and intervention strategies in residential buildings.
Authors achieve this using a "serious game approach" with a combination of direct user feedback using smart meters. 
The application also provides advice, comparisons, savings, reduction goals, and monitoring.
The approach takes into account almost all dimensions of residential energy usage. Their results show that their serious game was not
able to induce energy-saving behavior.

\cite{Jeong2021} used extreme points in the appliance usage curve to cluster usage profiles.
Usually, the first usage peak is in the morning, and the second one is in the evening. 
Additionally, they used demographic characteristics that are: region, area, age, salary, etc. to improve the results.
Using collected data, they clustered profiles. They had discovered 6 different usage profiles, 
where every cluster had a physical meaning such as energy-saving, morning heavy, evening heavy, etc.

Another clustering methodology was proposed by \cite{Park2019}, using load image profiles and image processing.
They represented time series data as an image. The image is a grid of squares where the y-axis contains monthly data with a resolution of one day,
x-axis contains daily data with a resolution of one hour. Grid is color filled with an algorithm that authors developed,
where red means more activity and blue less. Using digital image filters they transformed the type-1 image to type-2 and from there
used a threshold to obtain type-3. Using that information they clustered data based on images similarly. They used three different 
clustering methods: k-means, FCM, and EM algorithm. Using the Davies-Bouldin index, they were able to prove that image-based clustering performs better than non-image.

\cite{Joana2012} clustered different load profiles using electricity consumption data and surveys. They profiled residential homes. 
They used PCA and k-means resulting in 5 clusters. Similar to other load profiling papers. 

Whereas most of the above-mentioned papers focused on aggregated consumption of building to build a load profile,
authors \cite{Issi2018} focused on appliance-level load profiling.
Their main contribution was to create a realistic per appliance load profile.
They developed a wireless measurement system with smart plugs that enabled them to obtain 
power signatures for each appliance. They evaluated the data and based on observations they determined working cycles for each appliance.
Furthermore, they concluded that 15 \% of consumed power can be shifted, where they took tariffs into account. 

\section{Anomaly detection in building energy consumption data}

A review on Anomaly detection in building energy consumption data was written by \cite{HIMEUR2021116601}.
Here, the authors took a deep dive into detecting anomalies in energy consumption in buildings. 
The author first makes an overview of existing anomaly detection schemes and applications.
Second, they perform a critical analysis and an in-depth discussion of the state-of-the-art.
Next, they describe current trends such as NILM anomaly detection. Finally, they assemble a set of future research directions. 
Both reviews pointed out that NILM anomaly detection or NILM load profiling is a possible future research direction.

\cite{NILMAD2019} authors propose an algorithm
that functions on top of existing state-of-the-art NILM algorithms Hidden Markov model,
combinatorial optimization, Latent Bayesian Modeling, and Graph-based Signal Processing.
They focus on three appliances, a fridge, freezer, and heater. Their metric was the number of operation cycles and energy used within those cycles. 
They implemented sigma variables to represent standard deviation and used rule-based anomaly detection.
So if energy or counts are significantly larger than the mean then the day is considered anomalous.
Their rule had only one manual setting and that was a number of standard deviations before the sample was considered anomalous.
Their results show that sub-meter anomaly detection works decently whereas NILM-based anomaly does not work at all. 

\cite{NILMAD22019} published another paper in the same year, where they took a similar approach, except that they used 
only compressor-based appliances such as fridges and air conditioners. They also added a rule to their existing rule-based anomaly 
detection algorithm, but the results still showed that NILM algorithms are not there yet. 

\cite{Castangia2021} used disaggregated sub-meter data to detect anomalies in use consumption.
They used a private dataset of 20 homes from northern Italy with no synthetic anomalies. 
Dataset included data from 2018 to 2020 meaning it included covid-induced anomalies. 
The authors first pre-processed the data by aggregating input load in hourly energy consumption, 
the second derived additional features, which are the time of use and duration of the activation.
They use that data to detect single-point deviations for which they implemented the isolation Forest algorithm and
anomalous trends for which to detect, they implemented Change Point Detection. 


\section{Table of profiles}

While in related work I examined load profiling in general,
this chapter focuses on how data in load profiles is presented.  
It can be portrayed in various shapes and forms,
using all kinds of attributes and features to do so. 
First, main load profiling features will be defined.
Second, using these features a general load profile table will be constructed.
Third, references from related work and use cases will be mapped to the table.
Using this information main features will be selected.
Fourth, using a reduced feature set a more detailed table will be formed.
Again, the table will be populated using the same references as before.
Finally, using this information a research direction will be formed.

\subsection{Feature set} \label{sec:feature_set}

Using related work from this chapter and use-case references from chapter \ref{chapter4},
we can extract the most commonly used features to portray load profiles.

\begin{outline}
    \1 power
    \1 time
    \1 operating time (how long appliance or appliances is turned on)
    \1 appliances (a set)
    \1 Number of activations (How many times appliance or appliances were turned on)
\end{outline}

\subsection{General table}
Using these features we can form a table with all possible combinations.
Some combinations do not make logical sense and the others repeat themselves.
Such examples are marked with the letter X.
Table \ref{tab:general_map} is then populated with references from previous chapters.

To understand the table clearer, let's imagine that each feature is used as an axis label when plotting. 
That is for example why we did not use only the power consumption feature, but three variations of it. 
The first one is average power usage. It is an integer, and it could be an average of hourly, daily or yearly consumption. 
The next possibility is an array where we use a set of integers in any combination.
Finally, we have a histogram that is obtained from the power time array. 

\begin{table}[H]
    \caption{General table of load profiles}
    \label{tab:general_map}
    \begin{tabular}{|c|c|c|c|c|c|}
    \hline
        &
        frequency &
        appliances &
        \begin{tabular}[c]{@{}l@{}}number of\\ activations\end{tabular} &
        \begin{tabular}[c]{@{}l@{}}power\\ (avg)\end{tabular} &
        \begin{tabular}[c]{@{}l@{}}operating\\ time\end{tabular} \\ \hline
    appliances                                                      &   & X & X & X  & X    \\ \hline
    \begin{tabular}[c]{@{}c@{}}number of\\ activations\end{tabular} & X & \multicolumn{1}{c|}{\begin{tabular}[c]{@{}c@{}} \citeyear*{per_appliance_per_building} \\ \citeyear*{UKDALE} \end{tabular}} & X & X  & X    \\ \hline
    \begin{tabular}[c]{@{}c@{}}power \\ (avg)\end{tabular}          & X & \citeyear*{appliance_avgpower} &   & X  & X    \\ \hline
    \begin{tabular}[c]{@{}c@{}}power \\ (array)\end{tabular}        & \citeyear*{UKDALE} & X & X & X  & X    \\ \hline
    \begin{tabular}[c]{@{}c@{}}power \\ (histogram)\end{tabular}    &   &   & X & X  & X    \\ \hline
    \begin{tabular}[c]{@{}c@{}}operating\\ time\end{tabular}        & X & \citeyear*{operating_time} & \multicolumn{1}{c|}{\begin{tabular}[c]{@{}c@{}} \citeyear*{NILMAD2019} \\ \citeyear*{NILMAD22019} \\ \citeyear*{NILMAD2021} \end{tabular}} &  \citeyear*{NILMAD2021} & X    \\ \hline
    time array                                                      & X & X & \multicolumn{1}{c|}{\begin{tabular}[c]{@{}c@{}} \citeyear*{per_appliance_per_building} \\ \citeyear*{UKDALE} \end{tabular}} &  \multicolumn{1}{c|}{\begin{tabular}[c]{@{}c@{}} \citeyear*{Chuan2014} \\ \citeyear*{Csoknyai2019} \\ \citeyear*{H0} \\ \citeyear*{KAVOUSIAN2013184} \\ \citeyear*{WALKER1985} \\ 	\citeyear*{GERBEC2005} \\ 	\citeyear*{Gao2018} \\ 	\citeyear*{Jeong2021}\\  	\citeyear*{Joana2012} \\ 	\citeyear*{DER_heatmap_profile}\\ 	\citeyear*{NILMAD2019}\\	\citeyear*{NILMAD22019} \\	\citeyear*{Issi2018} \\	\citeyear*{NILMAD2021}\\	\citeyear*{Castangia2021} \\	\citeyear*{occupancy2013} \\	\citeyear*{Chuan2014} \\ 	\citeyear*{CAPASSO1994} \\ 	\citeyear*{Park2019} \\	\citeyear*{UKDALE} \\	\citeyear*{Gao2018} \end{tabular}}    & \citeyear*{OperatingTime_timeofday} \\ \hline
    \end{tabular}
\end{table}

The first profile is the one in the first row and the first column.
It is a combination of frequency and appliances. 
This would mean that we would have appliances on the x-axis and the number of occurrences o the y-axis.
The only useful application for this profile is when presenting how many appliances of one type are in one dataset. 
Other than that it has no use-cases and that is probably the reason why it is rarely used.

The next possible combination is between a number of activations and appliances. 
Here the explanation is relatively straightforward. The combination shows us how many times each appliance activities.

The second presentation of data is a combination of power and number of activations. 
In this case, we would construct a histogram of power values. 
On the x-axis we would have power values on the y-Axis we would have a number of times this power value occurred, where we would have buckets of a certain size.
Here we should not mix it up with the previous combination, since here we include the whole consumption and not only when turned on. 

The combination in the second column and second row is between average power, an integer, and appliances. Again explanation is simple. We present the average consumption for each appliance in a household.

The combination in the third column and second row, between average power and numbers of activations, is again less straightforward.
It presents how many times it activated with a certain power.

Here one feature is an output of the previous combination, a histogram of power. 
It is possible to have a histogram of the histogram, but it is not practical. 
This means we are looking at looking how many times did the occurrence occur. 
Not all combinations, even though they are possible, are useful. 

The combination in the second column and fifth row is again straightforward. 
Plotting histograms for each appliance and then plotting them side by side.
In this case, we present histograms in color, since we are working with 3 dimensions.

In the case of combining the second column and sixth row, between appliances and operating time, we present how long each appliance operates.
Here it could be either a number or even a time range presentation.
When doing a combination between operating time and the number of activations we are plotting how many times did an appliance turn on for a certain amount of time

In the case of a combination between operating time and power, we show how long did appliance operate with a certain power.

Next case, in the third row and last column, a combination between time array and several activations we present when did appliance turn on certain amount of times. 

We are coming to an end, and here we have the most commonly used case where we use time array and average power to present the data.

In the final case, the combination shows us at which time do appliance operate for a certain amount of time 

Based on table \ref{tab:general_map} it is possible to see that the most commonly
published feature combination is time and power. This combination will be used 
as a baseline when making a more detailed table. Although the operating time feature was 
explored in a few publications a bit, and it seems quite promising, we are 
focusing on activation-based histogram representation.
Based on table \ref{tab:general_map} it is possible to see that not much attention was given to it. 

\subsection{Detailed table}

This section will focus on exploring possible load profiles using activation-based histogram representation,
while using the power feature as a baseload. 
Features from \ref{tab:general_map} will be explored in higher detail. 
They will be split and arranged in a way that all 21 publications and power-based presentations will be divided into as many groups as possible. 
This should expose possible activation-based profiles as well as unpublished power-based profiles.

\subsubsection{Sub-features} \label{sec:subfeatures}

Main features were already described in section \ref{sec:feature_set}.
It is possible to further divide them into smaller so-called sub-features.
These are reshaped and grouped as follows:
\begin{outline}

\1 Way of presenting a profile
\2 Per-house load profile
\2 Per-appliance load profiles
\2 Per-house and per appliance load profile

\1 By time range of profile 
\2 Daily
\2 Weekly
\2 Monthly
\2 Yearly

\1 Way of measuring usage
\2 Average power use 
\2 Activation or frequency of activation
\end{outline}

\subsubsection{Sketches of load profiles}
A most common way load profiles are presented is a daily power consumption profile such as shown in figure \ref{fig:daily_power_profile}. 
The graph is a sketch, but it represents a standard load profile with morning and evening peaks.

\begin{figure}[H]
	\centering
	\caption{"Average daily usage profile for an appliance or a building"}
	\includegraphics[width=0.9\textwidth]{Figures/profile_sketches/Slide1.png}
	\label{fig:daily_power_profile}
\end{figure}

Some references include daily usage profiles as a histogram of activation at a point in a day, such as a figure \ref{fig:daily_act_profile}.

\begin{figure}[H]
	\centering
	\caption{"Histogram of daily activations profile for an appliance or a building"}
	\includegraphics[width=0.9\textwidth]{Figures/profile_sketches/Slide5.png}
	\label{fig:daily_act_profile}
\end{figure}

All figures can present whole-house usage or per-device usage. Each presentation has its pros and cons. 
To present more information sub-meter data can be used to represent whole-house usage with per-appliance contributions.
Such as on figure \ref{fig:daily_act_m_profile} and \ref{fig:daily_power_m_profile}.

\begin{figure}[H]
	\centering
	\caption{"Histogram of daily activations profile for an appliance A and B"}
	\includegraphics[width=0.9\textwidth]{Figures/profile_sketches/Slide8.png}
	\label{fig:daily_act_m_profile}
\end{figure}
\begin{figure}[H]
	\centering
	\caption{"Average weekday power consumption for appliances A, B and C"}
	\includegraphics[width=0.9\textwidth]{Figures/profile_sketches/Slide2.png}
	\label{fig:daily_power_m_profile}
\end{figure}

To present as much information as possible all above-mentioned attributes 
can be presented in a multidimensional way such as heatmap in a way shown in figure \ref{fig:heatmap_2dtime} and \ref{fig:heatmap_all_appl}.

\begin{figure}[H]
	\centering
	\caption{"Number of daily activations / power consumption of one appliance / house in one month period"}
	\includegraphics[width=0.9\textwidth]{Figures/profile_sketches/Slide10.png}
	\label{fig:heatmap_2dtime}
\end{figure}

Figure \ref{fig:heatmap_2dtime} presents a sketch, so the heatmap does not present the actual data. 
Even though, we can still see that the plot presents one month of data, where we can see the consumption throughout each day.
The brightness presents the activity of the household or even an appliance. 
The brighter the plot, the more activity for that hour of that day of the month.

One other thing to keep in mind when reading such a profile is that the origin is placed in the upper left corner.
This originates from image processing standards, where the origin of an image is in this location.
The main reason behind this practice is that arrays and matrices are written in such a way, and it just works better when wiring this as an image.

\begin{figure}[H]
	\centering
	\caption{"Number of activations / power consumption for each appliance in one month period"}
	\includegraphics[width=0.9\textwidth]{Figures/profile_sketches/Slide12.png}
	\label{fig:heatmap_all_appl}
\end{figure}

Instead of having two-time axes, we could use one axis for different appliances in a household. 
Such example is figure \ref{fig:heatmap_all_appl}.
In this case, we plot the consumption throughout the day on an x-axis, and a y-axis is used to display this for each appliance.

\subsection{Table of combinations or detailed table}

The above-shown profiles can be combined, yielding a new way of displaying the data.
Bellow, a map with combinations of the above-mentioned profiles is presented. 
The purpose of table \ref{fig:map_fig} to generate and show possible combinations.
Some combinations that had similar output were grouped, and some that could not be sketched were discarded. 

\begin{figure}[H]
	\centering
	\caption{"Table of combinations"}
	\includegraphics[width=0.9\textwidth]{Figures/profile_sketches/Slide14.png}
	\label{fig:map_fig}
\end{figure}

Figure above \ref{fig:map_fig}, uses sub-features from previous subsection \ref{sec:subfeatures}. 
In general, the table is formatted in a way that features from columns are
used in the x-axis of a plot, and rows are used in the y or z-axis of a plot. 

The column of the table presents the time domain. "Daily" means that the load profile presents
average usage for one day, weekly means it presents usage for a week.
To be clear, for one to construct a decent daily profile, one needs a few
weeks of data. The same goes for yearly profiles, in that case, one needs many years' worth of data. 

The top row of the table is composed of 3 main groups. 
The first group focuses on per-house energy consumption.
The second group examines the energy consumption of each appliance in a house separately.
Third group analyses all appliances in a building.

The next row of the table is further divided into two groups. First is the profile group
which presents the given usage unit on the y-axis and time on the x-axis. 
Next is a profile with a time group. 
In this case, we present the given usage unit on the z-axis and then time on the x and y-axis.
Here, the second-time dimension can be anything from a week to a year.
In the case of the per-house subgroup includes appliances instead of time. 
Example for this is figure \ref{fig:heatmap_all_appl}.
The last columns present the usage unit, that is power (P) or a number of activations (A).

\subsection{Mapping references to the table of profiles}

To find useful load profiles, references from related work must be mapped.

\begin{table}[H]
	\caption{Table presents previously mentioned load profiles}
	\label{tab:contributions}
	\begin{tabular}{|l|cccc|cccc|cccc|}
	\hline
	Description                                                   & \multicolumn{4}{c|}{Per-house}                                                                                               & \multicolumn{4}{c|}{Per-appliance}                                                                                           & \multicolumn{4}{c|}{\begin{tabular}[c]{@{}c@{}}Per-house\\ per-appliance\end{tabular}}                                                         \\ \hline
																 & \multicolumn{2}{c|}{LP}                         & \multicolumn{2}{c|}{\begin{tabular}[c]{@{}c@{}} + daily time \\ dimension \end{tabular}} & \multicolumn{2}{c|}{LP}                         & \multicolumn{2}{c|}{\begin{tabular}[c]{@{}c@{}} + daily time \\ dimension \end{tabular}} & \multicolumn{2}{c|}{LP}                               & \multicolumn{2}{c|}{\begin{tabular}[c]{@{}c@{}}Appliances\\ side by side\end{tabular}} \\ \hline
	\begin{tabular}[c]{@{}l@{}}Range of time\\ axis\end{tabular} & \multicolumn{1}{c|}{P} & \multicolumn{1}{c|}{A} & \multicolumn{1}{c|}{P}                         & A                         & \multicolumn{1}{c|}{P} & \multicolumn{1}{c|}{A} & \multicolumn{1}{c|}{P}                         & A                         & \multicolumn{1}{c|}{P}    & \multicolumn{1}{c|}{A}    & \multicolumn{1}{c|}{P}                               & A                               \\ \hline
	Daily                                                        & \multicolumn{1}{c|}{\begin{tabular}[c]{@{}c@{}} \citeyear*{UKDALE} \\ \citeyear*{Chuan2014} \\ \citeyear*{Csoknyai2019} \\ \citeyear*{H0} \\ \citeyear*{KAVOUSIAN2013184} \\ \citeyear*{CAPASSO1994} \\ \citeyear*{WALKER1985} \\ \citeyear*{GERBEC2005} \\ \citeyear*{Gao2018} \\ \citeyear*{Jeong2021} \\ \citeyear*{Joana2012} \\ \citeyear*{DER_heatmap_profile} \end{tabular}}  & \multicolumn{1}{c|}{}  & \multicolumn{1}{c|}{X}  &  X  & \multicolumn{1}{c|}{\begin{tabular}[c]{@{}l@{}} \citeyear*{NILMAD2019} \\ \citeyear*{NILMAD22019} \\ \citeyear*{Issi2018} \\ \citeyear*{NILMAD2021} \\ \citeyear*{Castangia2021} \\ \citeyear*{occupancy2013}	\end{tabular}}  & \multicolumn{1}{c|}{\citeyear*{UKDALE}}  & \multicolumn{1}{c|}{X}   &  \multicolumn{1}{c|}{X}   & \multicolumn{1}{c|}{\begin{tabular}[c]{@{}l@{}} \citeyear*{Chuan2014} \\ \citeyear*{CAPASSO1994} \\ \citeyear*{Gao2018} 	\end{tabular}}   & \multicolumn{1}{c|}{\citeyear*{UKDALE}}     & \multicolumn{1}{c|}{}      &    \\ \hline
	\begin{tabular}[c]{@{}l@{}}Weekly/\\ Monthly\end{tabular}    & \multicolumn{1}{c|}{\begin{tabular}[c]{@{}c@{}} \citeyear*{Csoknyai2019} \\ \citeyear*{H0} \\ \citeyear*{KAVOUSIAN2013184} \end{tabular}}  & \multicolumn{1}{c|}{}  & \multicolumn{1}{c|}{\begin{tabular}[c]{@{}l@{}} \citeyear*{2D_year_day_LP} \\ \citeyear*{Park2019} \\ \citeyear*{DER_heatmap_profile} \end{tabular}}                            &                           & \multicolumn{1}{c|}{}  & \multicolumn{1}{c|}{\citeyear*{per_appliance_per_building}}  & \multicolumn{1}{c|}{}                          &                           & \multicolumn{1}{c|}{\citeyear*{weekly_per_appliance_LP}}    & \multicolumn{1}{c|}{\citeyear*{per_appliance_per_building}}     & \multicolumn{1}{c|}{}                                &                                 \\ \hline
	Yearly                                                       & \multicolumn{1}{c|}{\begin{tabular}[c]{@{}c@{}} \citeyear*{Csoknyai2019} \\ \citeyear*{H0} \\ \citeyear*{KAVOUSIAN2013184} \end{tabular}}  & \multicolumn{1}{c|}{}  & \multicolumn{1}{c|}{}                          &                           & \multicolumn{1}{c|}{}  & \multicolumn{1}{c|}{}  & \multicolumn{1}{c|}{}                          &                           & \multicolumn{1}{c|}{}     & \multicolumn{1}{c|}{}     & \multicolumn{1}{c|}{}                                &                                 \\ \hline
	\end{tabular}
\end{table}

As can be seen from table \ref{tab:contributions}, most of the work (14 publications) has been done with standard daily load profiles with
per-house power usage such as figure \ref{fig:daily_power_profile}. 
Quite a lot of work (6 publications) has been done with per appliance daily power profiles.
A few publications were based on weekly and yearly load profiles,
and a few used two-dimensional time and power presentations.
Only one publication found used activation and time-based histogram such as 
shown in figure \ref{fig:daily_act_profile}. During the research we focused on publications
from minority classes, meaning not all existing publications for standard load profiles are included. 
The purpose of table \ref{tab:contributions} is to present missing scientific contributions and patterns of publications.  

\newcommand{\tabVar}{+ daily time \\ dimension  }
%\testIt

\subsection{Mapping use-cases to the table of combinations}

Table \ref{tab:use_cases} includes arranged publications from chapter \ref{Chapter5}. 
Similar pattern emerged as in table \ref{tab:contributions}. 

\begin{table}[H]
	\caption{Table presents references mentioned in use cases chapter}
	\label{tab:use_cases}
	\begin{tabular}{|l|cccc|cccc|cccc|}
	\hline
	Description &
	  \multicolumn{4}{c|}{Per-house} &
	  \multicolumn{4}{c|}{Per-appliance} &
	  \multicolumn{4}{c|}{\begin{tabular}[c]{@{}c@{}}Per-house\\ per-appliance\end{tabular}} \\  \hline
	  &
	  \multicolumn{2}{c|}{LP} &
	  \multicolumn{2}{c|}{\begin{tabular}[c]{@{}c@{}} \tabVar \end{tabular}} &
	  \multicolumn{2}{c|}{LP} &
	  \multicolumn{2}{c|}{\begin{tabular}[c]{@{}c@{}} \tabVar \end{tabular}} &
	  \multicolumn{2}{c|}{LP} &
	  \multicolumn{2}{c|}{\begin{tabular}[c]{@{}c@{}}Appliances\\ side by side\end{tabular}} \\ \hline
	\begin{tabular}[c]{@{}l@{}}Range of time\\ axis\end{tabular} &
	  \multicolumn{1}{c|}{P} &
	  \multicolumn{1}{c|}{A} &
	  \multicolumn{1}{c|}{P} &
	  A &
	  \multicolumn{1}{c|}{P} &
	  \multicolumn{1}{c|}{A} &
	  \multicolumn{1}{c|}{P} &
	  A &
	  \multicolumn{1}{c|}{P} &
	  \multicolumn{1}{c|}{A} &
	  \multicolumn{1}{c|}{P} &
	  A \\ \hline
	Daily &
	  \multicolumn{1}{c|}{\begin{tabular}[c]{@{}c@{}} \citeyear*{energy_saving1} \\ \citeyear*{energy_saving3} \\ \citeyear*{EV2020} \\ \citeyear*{energyStealing2018} \\ \citeyear*{shift2015} \\ \citeyear*{optimiseCostShift2015} \\ \citeyear*{controll2014} \end{tabular}} &
	  \multicolumn{1}{c|}{} &
	  \multicolumn{1}{c|}{X} &
    X
	   &
	  \multicolumn{1}{c|}{\begin{tabular}[c]{@{}c@{}} \citeyear*{EV2020} \\ \citeyear*{elder1} \\ \citeyear*{elder2} \\   \citeyear*{occupancy2013}	  \end{tabular}} &
	  \multicolumn{1}{c|}{} &
	  \multicolumn{1}{c|}{X} &
    X
	   &
	  \multicolumn{1}{c|}{\citeyear*{Chuan2014}	 } &
	  \multicolumn{1}{c|}{} &
	  \multicolumn{1}{c|}{} &
	   \\ \hline
	\begin{tabular}[c]{@{}l@{}}Weekly/\\ Monthly\end{tabular} &
	  \multicolumn{1}{c|}{\begin{tabular}[c]{@{}c@{}} \citeyear*{energy_saving3} \\ \citeyear*{KAVOUSIAN2013184}  \end{tabular}} &
	  \multicolumn{1}{c|}{} &
	  \multicolumn{1}{c|}{} &
	   &
	  \multicolumn{1}{c|}{} &
	  \multicolumn{1}{c|}{} &
	  \multicolumn{1}{c|}{} &
	   &
	  \multicolumn{1}{c|}{} &
	  \multicolumn{1}{c|}{} &
	  \multicolumn{1}{c|}{} &
	   \\ \hline
	Yearly &
	  \multicolumn{1}{c|}{\begin{tabular}[c]{@{}c@{}}\citeyear*{energy_saving3}\end{tabular}} &
	  \multicolumn{1}{c|}{} &
	  \multicolumn{1}{c|}{} &
	   &
	  \multicolumn{1}{c|}{} &
	  \multicolumn{1}{c|}{} &
	  \multicolumn{1}{c|}{} &
	   &
	  \multicolumn{1}{c|}{} &
	  \multicolumn{1}{c|}{} &
	  \multicolumn{1}{c|}{} &
	   \\ \hline
	\end{tabular}
\end{table}

\subsection{Table of use case groups}

The table \ref{tab:groups} presents same publications as \ref{tab:use_cases},
but only group names are shown.
The table indicates how groups are arranged.
Where anomaly detection and elderly care are dominating in the per-appliance part of the table,
energy-saving and grid management are dominating in a per-house part of the table. 

\begin{table}[H]
    \caption{Table presents references mentioned in use cases chapter}
	\label{tab:groups}
    \begin{adjustbox}{width=1.2\textwidth,center=\textwidth} 
        \begin{tabular}{|l|cccc|cccc|cccc|}
            \hline
            \begin{tabular}[c]{@{}l@{}}ES - energy saving\\ GM - grid management\\ AD - anomaly detection\\ EC - elderly care\\ X - unfeasible\end{tabular} &
              \multicolumn{4}{c|}{Per-house} &
              \multicolumn{4}{c|}{Per-appliance} &
              \multicolumn{4}{c|}{\begin{tabular}[c]{@{}c@{}}Per-house\\ per-appliance\end{tabular}} \\ \cline{2-13} 
             &
              \multicolumn{2}{c|}{LP} &
              \multicolumn{2}{c|}{\begin{tabular}[c]{@{}c@{}}+ daily time \\ dimension\end{tabular}} &
              \multicolumn{2}{c|}{LP} &
              \multicolumn{2}{c|}{\begin{tabular}[c]{@{}c@{}} \tabVar \end{tabular}} &
              \multicolumn{2}{c|}{LP} &
              \multicolumn{2}{c|}{\begin{tabular}[c]{@{}c@{}}Appliances\\ side by side\end{tabular}} \\ \hline
            \begin{tabular}[c]{@{}l@{}}Range of time\\ axis\end{tabular} &
              \multicolumn{1}{c|}{P} &
              \multicolumn{1}{c|}{A} &
              \multicolumn{1}{c|}{P} &
              A &
              \multicolumn{1}{c|}{P} &
              \multicolumn{1}{c|}{A} &
              \multicolumn{1}{c|}{P} &
              A &
              \multicolumn{1}{c|}{P} &
              \multicolumn{1}{c|}{A} &
              \multicolumn{1}{c|}{P} &
              A \\ \hline
            Daily &
              \multicolumn{1}{c|}{ES, GM} &
              \multicolumn{1}{c|}{} &
              \multicolumn{1}{c|}{X} &
              X
               &
              \multicolumn{1}{c|}{\begin{tabular}[c]{@{}c@{}}AD, EC,\\ ES\end{tabular}} &
              \multicolumn{1}{c|}{} &
              \multicolumn{1}{c|}{X} &
              X
               &
              \multicolumn{1}{c|}{GM} &
              \multicolumn{1}{c|}{} &
              \multicolumn{1}{c|}{} &
               \\ \hline
            \begin{tabular}[c]{@{}l@{}}Weekly/\\ Monthly\end{tabular} &
              \multicolumn{1}{c|}{ES} &
              \multicolumn{1}{c|}{} &
              \multicolumn{1}{c|}{} &
               &
              \multicolumn{1}{c|}{} &
              \multicolumn{1}{c|}{} &
              \multicolumn{1}{c|}{} &
               &
              \multicolumn{1}{c|}{} &
              \multicolumn{1}{c|}{} &
              \multicolumn{1}{c|}{} &
               \\ \hline
            Yearly &
              \multicolumn{1}{c|}{ES} &
              \multicolumn{1}{c|}{} &
              \multicolumn{1}{c|}{} &
               &
              \multicolumn{1}{c|}{} &
              \multicolumn{1}{c|}{} &
              \multicolumn{1}{c|}{} &
               &
              \multicolumn{1}{c|}{} &
              \multicolumn{1}{c|}{} &
              \multicolumn{1}{c|}{} &
               \\ \hline
            \end{tabular}
    \end{adjustbox} 
    \end{table}

The figures listed above clearly depict the void not filled by publications. 
Although they may not be published, they still have a possible use case. 
In table \ref{tab:groups_proposed} empty spaces are filled 
with possible use cases for given load profiles. 

\begin{table}[H]
    \caption{"Proposed use cases for profiles"}
    \label{tab:groups_proposed}
    \begin{adjustbox}{width=1.2\textwidth,center=\textwidth} 
    \begin{tabular}{|l|cccc|cccc|cccc|}
    \hline
    \begin{tabular}[c]{@{}l@{}}ES - energy saving\\ GM - grid management\\ AD - anomaly detection\\ EC - elderly care\\ X - unfeasible\end{tabular} &
      \multicolumn{4}{c|}{Per-house} &
      \multicolumn{4}{c|}{Per-appliance} &
      \multicolumn{4}{c|}{\begin{tabular}[c]{@{}c@{}}Per-house\\ per-appliance\end{tabular}} \\ \cline{2-13} 
     &
      \multicolumn{2}{c|}{LP} &
      \multicolumn{2}{c|}{\begin{tabular}[c]{@{}c@{}}+ daily time \\ dimension\end{tabular}} &
      \multicolumn{2}{c|}{LP} &
      \multicolumn{2}{c|}{\begin{tabular}[c]{@{}c@{}} \tabVar \end{tabular}} &
      \multicolumn{2}{c|}{LP} &
      \multicolumn{2}{c|}{\begin{tabular}[c]{@{}c@{}}Appliances\\ side by side\end{tabular}} \\ \hline
    \begin{tabular}[c]{@{}l@{}}Range of time\\ axis\end{tabular} &
      \multicolumn{1}{c|}{P} &
      \multicolumn{1}{c|}{A} &
      \multicolumn{1}{c|}{P} &
      A &
      \multicolumn{1}{c|}{P} &
      \multicolumn{1}{c|}{A} &
      \multicolumn{1}{c|}{P} &
      A &
      \multicolumn{1}{c|}{P} &
      \multicolumn{1}{c|}{A} &
      \multicolumn{1}{c|}{P} &
      A \\ \hline
    Daily &
      \multicolumn{1}{c|}{\begin{tabular}[c]{@{}c@{}}AD,\\ ES, GM,\end{tabular}} &
      \multicolumn{1}{c|}{\begin{tabular}[c]{@{}c@{}}AD,\\ ES, GM,\end{tabular}} &
      \multicolumn{1}{c|}{X} &
      X &
      \multicolumn{1}{c|}{\begin{tabular}[c]{@{}c@{}}AD, EC,\\ ES, GM\end{tabular}} &
      \multicolumn{1}{c|}{\begin{tabular}[c]{@{}c@{}}AD, EC,\\ ES, GM\end{tabular}} &
      \multicolumn{1}{c|}{\begin{tabular}[c]{@{}c@{}} X \end{tabular}} &
      \begin{tabular}[c]{@{}c@{}} X \end{tabular} &
      \multicolumn{1}{c|}{\begin{tabular}[c]{@{}c@{}}AD, EC,\\ ES, GM\end{tabular}} &
      \multicolumn{1}{c|}{\begin{tabular}[c]{@{}c@{}}AD, EC,\\ ES, GM\end{tabular}} &
      \multicolumn{1}{c|}{\begin{tabular}[c]{@{}c@{}}AD, EC,\\ ES, GM\end{tabular}} &
      \begin{tabular}[c]{@{}c@{}}AD, EC,\\ ES, GM\end{tabular} \\ \hline
    \begin{tabular}[c]{@{}l@{}}Weekly/\\ Monthly\end{tabular} &
      \multicolumn{1}{c|}{\begin{tabular}[c]{@{}c@{}}AD,\\ ES, GM\end{tabular}} &
      \multicolumn{1}{c|}{\begin{tabular}[c]{@{}c@{}}AD,\\ ES, GM,\end{tabular}} &
      \multicolumn{1}{c|}{ES, GM} &
      ES, GM &
      \multicolumn{1}{c|}{\begin{tabular}[c]{@{}c@{}}AD,\\ ES, GM\end{tabular}} &
      \multicolumn{1}{c|}{\begin{tabular}[c]{@{}c@{}}AD,\\ ES, GM\end{tabular}} &
      \multicolumn{1}{c|}{\begin{tabular}[c]{@{}c@{}}AD,\\ ES, GM\end{tabular}} &
      \begin{tabular}[c]{@{}c@{}}AD,\\ ES, GM\end{tabular} &
      \multicolumn{1}{c|}{\begin{tabular}[c]{@{}c@{}}AD,\\ ES, GM\end{tabular}} &
      \multicolumn{1}{c|}{\begin{tabular}[c]{@{}c@{}}AD,\\ ES, GM\end{tabular}} &
      \multicolumn{1}{c|}{\begin{tabular}[c]{@{}c@{}}AD,\\ ES, GM\end{tabular}} &
      \begin{tabular}[c]{@{}c@{}}AD,\\ ES, GM\end{tabular} \\ \hline
    Yearly &
      \multicolumn{1}{c|}{ES, GM} &
      \multicolumn{1}{c|}{ES, GM} &
      \multicolumn{1}{c|}{ES, GM} &
      ES, GM &
      \multicolumn{1}{c|}{\begin{tabular}[c]{@{}c@{}}AD,\\ ES, GM\end{tabular}} &
      \multicolumn{1}{c|}{\begin{tabular}[c]{@{}c@{}}AD,\\ ES, GM\end{tabular}} &
      \multicolumn{1}{c|}{ ES, GM } &
      ES, GM &
      \multicolumn{1}{c|}{ES,GM}  &
      \multicolumn{1}{c|}{\begin{tabular}[c]{@{}c@{}}AD,\\ ES, GM\end{tabular}} &
      \multicolumn{1}{c|}{\begin{tabular}[c]{@{}c@{}}AD,\\ ES, GM\end{tabular}} &
      \begin{tabular}[c]{@{}c@{}}AD,\\ ES, GM\end{tabular} \\ \hline
    \end{tabular}
    \end{adjustbox}
\end{table}

\subsection{Table of load profile potentials} \label{subsec:potential}

Some combinations are indeed illogical and again others are less useful in a practical sense.
The next table will try to rate the scientific potential of the profiles based on two characteristics. 
First is how well data is presented to the user,
meaning that the load profile is clear at what it is presenting.
The second is the effectiveness when being used in an algorithm, or in other words, how well data is presented to a machine. 
These characteristics can not be easily measured,
but it is possible to extract them based on the pattern of publications.
To do that, we have to make two assumptions.
The first one would be, that the larger the number of publications, the larger is the effectiveness of presenting the data to a human.
The second would be, that the larger the number of use cases, the better the effectiveness of presenting the data to a machine.
Using these two assumptions, we propose the following table. 
The table has four possible classes. 

\begin{outline} 
\1 1 - The load profile satisfies both assumptions and has a high utility rate and was already researched (high research potential). 
\1 2 - The load profile satisfies only one of the above-mentioned assumptions (mid-research potential).
\1 3 - The load profile does not suffice any of the above-mentioned assumptions and was not yet researched or practically used (low research potential).
\1 X - The load profile is inexplicable.
\end{outline}

\begin{table}[H]
    \caption{Proposed classification of profiles}
    \label{tab:classified_profiles}
    \begin{tabular}{|l|cccc|cccc|cccc|}
    \hline
     &
      \multicolumn{4}{c|}{Per-house} &
      \multicolumn{4}{c|}{Per-appliance} &
      \multicolumn{4}{c|}{\begin{tabular}[c]{@{}c@{}}Per-house\\ per-appliance\end{tabular}} \\ \cline{2-13} 
     &
      \multicolumn{2}{c|}{LP} &
      \multicolumn{2}{c|}{\begin{tabular}[c]{@{}c@{}}+ daily time \\ dimension\end{tabular}} &
      \multicolumn{2}{c|}{LP} &
      \multicolumn{2}{c|}{\begin{tabular}[c]{@{}c@{}} \tabVar \end{tabular}} &
      \multicolumn{2}{c|}{LP} &
      \multicolumn{2}{c|}{\begin{tabular}[c]{@{}c@{}}Appliances\\ side by side\end{tabular}} \\ \hline
    \begin{tabular}[c]{@{}l@{}}Range of time\\ axis\end{tabular} &
      \multicolumn{1}{c|}{P} &
      \multicolumn{1}{c|}{A} &
      \multicolumn{1}{c|}{P} &
      A &
      \multicolumn{1}{c|}{P} &
      \multicolumn{1}{c|}{A} &
      \multicolumn{1}{c|}{P} &
      A &
      \multicolumn{1}{c|}{P} &
      \multicolumn{1}{c|}{A} &
      \multicolumn{1}{c|}{P} &
      A \\ \hline
    Daily &
      \multicolumn{1}{c|}{1} &
      \multicolumn{1}{c|}{3} &
      \multicolumn{1}{c|}{X} &
      X &
      \multicolumn{1}{c|}{1} &
      \multicolumn{1}{c|}{2} &
      \multicolumn{1}{c|}{X} &
      X &
      \multicolumn{1}{c|}{1} &
      \multicolumn{1}{c|}{2} &
      \multicolumn{1}{c|}{3} &
      3 \\ \hline
    \begin{tabular}[c]{@{}l@{}}Weekly/\\ Monthly\end{tabular} &
      \multicolumn{1}{c|}{1} &
      \multicolumn{1}{c|}{3} &
      \multicolumn{1}{c|}{2} &
      3 &
      \multicolumn{1}{c|}{3} &
      \multicolumn{1}{c|}{2} &
      \multicolumn{1}{c|}{3} &
      3 &
      \multicolumn{1}{c|}{2} &
      \multicolumn{1}{c|}{2} &
      \multicolumn{1}{c|}{3} &
      3 \\ \hline
    Yearly &
      \multicolumn{1}{c|}{1} &
      \multicolumn{1}{c|}{3} &
      \multicolumn{1}{c|}{3} &
      3 &
      \multicolumn{1}{c|}{3} &
      \multicolumn{1}{c|}{3} &
      \multicolumn{1}{c|}{3} &
      3 &
      \multicolumn{1}{c|}{3} &
      \multicolumn{1}{c|}{3} &
      \multicolumn{1}{c|}{3} &
      3 \\ \hline
    \end{tabular}
\end{table}

\subsection{Table of possible future research directions}

The table \ref{tab:classified_profiles} shows which profiles were researched and which were not yet.
To find future research directions we must look into profiles that were least researched.
These profiles are marked with the number 3 on table \ref{tab:classified_profiles}.
Some profiles were not researched because they may not present data as well,
and some were simply overlooked. 

This is why we have built the following table \ref{tab:future_rd}.
The table was marked as follows:

\begin{outline} 
\1 (1) - The load profile has high potential. 
\1 (2) - The load profile has mid-potential.
\1 Empty - The load profile has low potential or was already researched.
\1 X - load profile is inexplicable
\end{outline}

The process of evaluation was a bit complicated, but it can be summed down to the following rules.

If the load profile was used as a power profile, can it be used as an activation profile?
Here we must use some common sense since per-building power load profiles were commonly used and based on that activation load profiles should be useful as well.
That is not the case since to build per-building activation load profiles we need per-appliance (sub-meter) data anyway. That is why we have assigned them to the second class.

Another rule that was followed was useful for combined load profiles, such as 2D time profiles. 
If one dimension of load profiles was used, with what other dimension could it be combined?
In the case where one dimension was commonly used, it is probably worth investigating it with a combination of additional dimensions. 

Following these rules, the table  \ref{tab:future_rd} was constructed.


\begin{table}[H]
    \caption{Possible future research contributions}
    \label{tab:future_rd}
    \begin{tabular}{|l|cccc|cccc|cccc|}
    \hline
     &
      \multicolumn{4}{c|}{Per-house} &
      \multicolumn{4}{c|}{Per-appliance} &
      \multicolumn{4}{c|}{\begin{tabular}[c]{@{}c@{}}Per-house\\ per-appliance\end{tabular}} \\ \cline{2-13} 
     &
      \multicolumn{2}{c|}{LP} &
      \multicolumn{2}{c|}{\begin{tabular}[c]{@{}c@{}}+ daily time \\ dimension\end{tabular}} &
      \multicolumn{2}{c|}{LP} &
      \multicolumn{2}{c|}{\begin{tabular}[c]{@{}c@{}} \tabVar \end{tabular}} &
      \multicolumn{2}{c|}{LP} &
      \multicolumn{2}{c|}{\begin{tabular}[c]{@{}c@{}}Appliances\\ side by side\end{tabular}} \\ \hline
    \begin{tabular}[c]{@{}l@{}}Range of time\\ axis\end{tabular} &
      \multicolumn{1}{c|}{P} &
      \multicolumn{1}{c|}{A} &
      \multicolumn{1}{c|}{P} &
      A &
      \multicolumn{1}{c|}{P} &
      \multicolumn{1}{c|}{A} &
      \multicolumn{1}{c|}{P} &
      A &
      \multicolumn{1}{c|}{P} &
      \multicolumn{1}{c|}{A} &
      \multicolumn{1}{c|}{P} &
      A \\ \hline
    Daily &
      \multicolumn{1}{c|}{} &
      \multicolumn{1}{c|}{(2)} &
      \multicolumn{1}{c|}{X} &
      X
       &
      \multicolumn{1}{c|}{} &
      \multicolumn{1}{c|}{} &
      \multicolumn{1}{c|}{X} &
      X
       &
      \multicolumn{1}{c|}{} &
      \multicolumn{1}{c|}{} &
      \multicolumn{1}{c|}{(1)} &
      (1) \\ \hline
    \begin{tabular}[c]{@{}l@{}}Weekly/\\ Monthly\end{tabular} &
      \multicolumn{1}{c|}{} &
      \multicolumn{1}{c|}{(2)} &
      \multicolumn{1}{c|}{} &
      (1) &
      \multicolumn{1}{c|}{(1)} &
      \multicolumn{1}{c|}{} &
      \multicolumn{1}{c|}{(1)} &
      (1) &
      \multicolumn{1}{c|}{} &
      \multicolumn{1}{c|}{} &
      \multicolumn{1}{c|}{(2)} &
      (2) \\ \hline
    Yearly &
      \multicolumn{1}{c|}{} &
      \multicolumn{1}{c|}{} &
      \multicolumn{1}{c|}{} &
       &
      \multicolumn{1}{c|}{(2)} &
      \multicolumn{1}{c|}{(2)} &
      \multicolumn{1}{c|}{} &
       &
      \multicolumn{1}{c|}{} &
      \multicolumn{1}{c|}{} &
      \multicolumn{1}{c|}{(2)} &
      (2) \\ \hline
    \end{tabular}
\end{table}

Table \ref{tab:future_rd} presents load profiles that we will pursue in 
this paper. We will focus on profiles from the first class and activation
frequency type of usage. When the aforementioned parameters are applied, 
the result is table \ref{tab:our_rd}

\begin{table}[H]
    \caption{Load profiles to be pursued }
    \label{tab:our_rd}
    \begin{tabular}{|l|cccc|cccc|cccc|}
    \hline
     &
      \multicolumn{4}{c|}{Per-house} &
      \multicolumn{4}{c|}{Per-appliance} &
      \multicolumn{4}{c|}{\begin{tabular}[c]{@{}c@{}}Per-house\\ per-appliance\end{tabular}} \\ \cline{2-13} 
     &
      \multicolumn{2}{c|}{LP} &
      \multicolumn{2}{c|}{\begin{tabular}[c]{@{}c@{}}+ daily time \\ dimension\end{tabular}} &
      \multicolumn{2}{c|}{LP} &
      \multicolumn{2}{c|}{\begin{tabular}[c]{@{}c@{}} \tabVar \end{tabular}} &
      \multicolumn{2}{c|}{LP} &
      \multicolumn{2}{c|}{\begin{tabular}[c]{@{}c@{}}Appliances\\ side by side\end{tabular}} \\ \hline
    \begin{tabular}[c]{@{}l@{}}Range of time\\ axis\end{tabular} &
      \multicolumn{1}{c|}{P} &
      \multicolumn{1}{c|}{A} &
      \multicolumn{1}{c|}{P} &
      A &
      \multicolumn{1}{c|}{P} &
      \multicolumn{1}{c|}{A} &
      \multicolumn{1}{c|}{P} &
      A &
      \multicolumn{1}{c|}{P} &
      \multicolumn{1}{c|}{A} &
      \multicolumn{1}{c|}{P} &
      A \\ \hline
    Daily &
      \multicolumn{1}{c|}{} &
      \multicolumn{1}{c|}{} &
      \multicolumn{1}{c|}{X} &
      X
      &
      \multicolumn{1}{c|}{} &
      \multicolumn{1}{c|}{} &
      \multicolumn{1}{c|}{X} &
      X
      &
      \multicolumn{1}{c|}{} &
      \multicolumn{1}{c|}{} &
      \multicolumn{1}{c|}{} &
      (1) \\ \hline
    \begin{tabular}[c]{@{}l@{}}Weekly/\\ Monthly\end{tabular} &
      \multicolumn{1}{c|}{} &
      \multicolumn{1}{c|}{} &
      \multicolumn{1}{c|}{} &
      (1) &
      \multicolumn{1}{c|}{} &
      \multicolumn{1}{c|}{} &
      \multicolumn{1}{c|}{} &
      (1) &
      \multicolumn{1}{c|}{} &
      \multicolumn{1}{c|}{} &
      \multicolumn{1}{c|}{} &
       \\ \hline
    Yearly &
      \multicolumn{1}{c|}{} &
      \multicolumn{1}{c|}{} &
      \multicolumn{1}{c|}{} &
       &
      \multicolumn{1}{c|}{} &
      \multicolumn{1}{c|}{} &
      \multicolumn{1}{c|}{} &
       &
      \multicolumn{1}{c|}{} &
      \multicolumn{1}{c|}{} &
      \multicolumn{1}{c|}{} &
       \\ \hline
    \end{tabular}
    \end{table}


Based on the table \ref{tab:our_rd} we will use the following profiles in the following chapters.

In chapter 7 we will use 
\begin{itemize}
  \item Per-house daily-weekly load profile
  \item Per-appliance daily-weekly load profile
\end{itemize}
with a t-SNE neighboring algorithm to find how they are related in high dimensional space.

In chapter 8 we will use
\begin{itemize}
  \item Per-house Per-appliance daily load profiles with appliances side by side
\end{itemize}
To build assisted living system for the elderly.

