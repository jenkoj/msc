\chapter{Conclusion}
\label{chapter7}

In the introduction of Chapter \ref{chapter1}, it was said that the goal of the thesis will be achieved by contributing the following:

\begin{enumerate}
	\item Surveying the state-of-the-art LPs (Chapter \ref{chapter2})
	\item Development of multidimensional activation LPs (Chapter \ref{chapter4})
	\item Exploratory data analysis of activation LP's through t-SNE (Chapter \ref{chapter5})
	\item Propose a new anomaly detection method for elderly care (Chapter \ref{chapter6})
\end{enumerate}

With the first contribution, we have found new, previously unused ways of presenting the data.
This was achieved by building a detailed table of profiles such as we have seen in Chapter \ref{chapter2}.
This table presented the missing gaps, and which presentations were not used by the community.
We knew that not all unused profiles were useful, by using other publications we classified them based on their impact. 
We have selected the few with the highest impact and utilized used them in the following chapters.

Furthermore, we presented all the LPs in high detail. This was done so that the reader was able to understand what the LPs look like and what they present.
While doing so, we pointed out how some profiles could be used, and how we will use them to prove that they are useful. 

The third was contributed in Chapter \ref{chapter5}, where we have shown how data is connected in high-dimension space
using t-SNE for dimensionality reduction. Here we have shown how some buildings have more similar activation patterns than others.
Furthermore, we have shown which appliances are being used similarly. 
We have grouped the appliances into appliance groups and showed that appliances from different datasets are being used similarly, and how this method and groups can help us label unlabeled data. 
The formed clusters showed that a routine and persistent usage pattern does exist. This laid the groundwork for elderly care, where we have used this routine at the center of the algorithm.

The last was contributed in Chapter \ref{chapter6}, by building functioning elderly care assisted living system. 
The results proved that we successfully used one of the proposed LPs in a real-world scenario. 
The main goal was to efficiently extract the routine, and build a working system around it.
The results show that we have succeeded in doing so and that the algorithm is adequate to be used in the real world.
To further prepare the algorithm for the real world, we have implemented an iterative learning system.
The system could be put online a month after the installation of the system and continues to improve over time.
Furthermore, we have shown that a correlation between t-SNE euclidean distance and calculated routine rate does exist,
which proves our statement from Chapter \ref{chapter5} that formed clusters present routine.

We believe that our work has contributed new tools for understanding energy consumption and leveraging energy data as a passive sensor to detect anomalies in the consumption itself.
Energy efficiency and the improvement of the well-being of the elderly are just two use-cases we addressed.
There is still much more to learn about how and which LPs can be used to improve our lives.
While we have filled in a few gaps in the table of profiles, it is up to scientific community to fill in the rest.

% [We could go into larger details and use other dimensionality reduction algorithms for comparison, or even have more empirical proofs. 
% In the case of elderly care, we could use the results or algorithms of other publications to compare it to ours, or even compare it to the other intrusive methods. 
% This could all be done in greater detail.

% In the end, that was not the goal.
% The goal of the thesis was to prove that the proposed profiles can be efficiently utilized. 
% By doing that, we have achieved the main goal of the thesis,
% that was "to propose suitable consumption profiles for supporting residential building consumption optimization and elderly care management".
% With that, we can conclude the thesis with the following words:

% The ever-increasing amount of data is available to the scientific community.
% This data can be fully utilized if we find ways to efficiently extract the information that it is holding.
% The sole purpose of the LPs is to reveal patterns, contextual features and information itself in the vast sea of data.
% With the proposed LPs, we have hopefully contributed new tools that will help researchers to uncover the truths that the datasets hold. 
% While we have filled in a few gaps in the table of profiles, it is up to scientists community to fill in the rest.]