\chapter{Conclusion}
\label{chapter7}

In the introduction of Chapter \ref{chapter1}, we stated that the goal of the thesis would be achieved by contributing the following:

\begin{enumerate}
\item Surveying the state-of-the-art LPs (Chapter \ref{chapter2})
\item Developing multidimensional activation LPs (Chapter \ref{chapter4})
\item Conducting exploratory data analysis of activation LPs through t-SNE (Chapter \ref{chapter5})
\item Proposing a new anomaly detection method for elderly care (Chapter \ref{chapter6})
\end{enumerate}

With the first contribution, we have found new, previously unused ways of presenting the data.
This was achieved by building a detailed table of profiles such as we have seen in Chapter \ref{chapter2}.
This table presented the missing gaps, and which presentations were not used by the community.
We knew that not all unused profiles were useful, by using other publications we classified them based on their impact. 
We have selected the few with the highest impact and used them in the following chapters.

Furthermore, we presented all the LPs in high detail, enabling the reader to understand what the LPs look like and what they represent. 
While doing so, we discussed possible use-cases for the LPs and showcased our plans on how we intend to utilize some of the previously unused activation LPs.

The third contribution was made in Chapter \ref{chapter5}, where we have shown how data is connected in high-dimensional space
using t-SNE for dimensionality reduction. Here, we show that some buildings have more similar activation patterns than others.
Furthermore, we identified which appliances are used similarly.
We grouped the appliances into appliance groups and showed that appliances from different datasets are used similarly, and how this method and groups can help us label unlabeled data.
The formed clusters revealed that a routine and persistent usage pattern does indeed exist. This laid the groundwork for elderly care, where we have used this routine at the center of the algorithm.

The last contribution was made in Chapter \ref{chapter6}, by building functioning elderly care assisted living system. 
The results confirmed that we successfully used one of the proposed LPs in a real-world scenario. 
The main goal was to efficiently extract the routine and then build a working system around it.
The results show that we have succeeded in doing so and that the algorithm is adequate for use in the real world.
To further prepare the algorithm for the real world, we have implemented an iterative learning system.
The system could be put online a month after the installation, and it continues to improve over time.
Furthermore, we have shown a correlation between t-SNE Euclidean distance and the calculated routine rate,
which proves our statement from Chapter \ref{chapter5} that formed clusters represent routine.

The sole purpose of the load profiles is to reveal patterns, contextual features and information itself in the vast sea of data.
With the proposed load profiles, we have hopefully contributed new tools that will help researchers to uncover the truths held within data.
We have applied these tools to improve the independence and well-being of the elderly and discussed how they can be used in numerous use-cases one of them being energy efficiency.
There is still much more to learn about how and which LPs can be used to improve our lives, and the proposed table of profiles can serve as a good starting point for such research.
While we have filled in a few gaps in the table of profiles, it is up to the scientific community to fill in the rest.

% [We could go into larger details and use other dimensionality reduction algorithms for comparison, or even have more empirical proofs. 
% In the case of elderly care, we could use the results or algorithms of other publications to compare it to ours, or even compare it to the other intrusive methods. 
% This could all be done in greater detail.

% In the end, that was not the goal.
% The goal of the thesis was to prove that the proposed profiles can be efficiently utilized. 
% By doing that, we have achieved the main goal of the thesis,
% that was "to propose suitable consumption profiles for supporting residential building consumption optimization and elderly care management".
% With that, we can conclude the thesis with the following words:

% The ever-increasing amount of data is available to the scientific community.
% This data can be fully utilized if we find ways to efficiently extract the information that it is holding.
% The sole purpose of the LPs is to reveal patterns, contextual features and information itself in the vast sea of data.
% With the proposed LPs, we have hopefully contributed new tools that will help researchers to uncover the truths that the datasets hold. 
% While we have filled in a few gaps in the table of profiles, it is up to scientists community to fill in the rest.]